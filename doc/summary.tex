\section{Podsumowanie} 

Na otrzymanych wykresach widać, że jeżeli po 60 iteracjach wyliczone wagi zbiegają, to zbiegają do podobnych wartości:
 \begin{itemize}
 \item PiecesNumberStrategy - ~0.5 lub ~3.9,
 \item MobilityStrategy - w przedziale ~6.0 do ~8.0,
 \item CornerStrategy - ~5.0,
 \item EdgesStrategy - ~4.0,
 \item NearCornerStrategy - ~2.0,
 \item NearEdgesStrategy - ~0.5.
\end{itemize}

Najprawdopodobniej MobilityStrategy jest najważniejszą strategią (zawsze otrzymuje dużą wagę). Kolejne w kolejności są strategie CornerStrategy i EdgesStrategy, co potwierdza wcześniejsze przypuszczenia. Mało istotne w trakcie gry są miejsca koło krawędzi i rogów. Łatwo również zauważyć, że strategia PiecesNumberStrategy często jest mało istotna ze względu na brak jej efektywności. Potwierdza to stwierdzenie, że w othello istotny jest rozkład pionów na planszy, niekoniecznie zaś ich liczba.