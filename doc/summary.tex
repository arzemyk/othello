\section{Podsumowanie} 

Na otrzymanych wykresach widać, że jeżeli po 60 iteracjach wyliczone wagi zbiegają, to zbiegają do podobnych wartości:
 \begin{itemize}
 \item PiecesNumberStrategy - $\sim0.5$ lub $\sim3.9$,
 \item MobilityStrategy - w przedziale $\sim6.0$ do $\sim8.0$,
 \item CornerStrategy - $\sim5.0$,
 \item EdgesStrategy - $\sim4.0$,
 \item NearCornerStrategy - $\sim2.0$,
 \item NearEdgesStrategy - $\sim0.5$.
\end{itemize}

Najprawdopodobniej MobilityStrategy jest najlepszą strategią (zawsze otrzymuje dużą wagę). Kolejne w kolejności są strategie CornerStrategy i EdgesStrategy, co potwierdza wcześniejsze przypuszczenia. Mało istotne w trakcie gry są miejsca koło krawędzi i rogów. Łatwo również zauważyć, że strategia PiecesNumberStrategy często jest mało istotna ze względu na brak jej efektywności. Potwierdza to stwierdzenie, że w Othello istotny jest rozkład pionów na planszy, niekoniecznie zaś ich liczba.